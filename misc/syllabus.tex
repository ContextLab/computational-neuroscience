\documentclass[12pt]{article}
\textwidth=6.7in
\textheight=8in
\topmargin=-0.5in
\headheight=0in
\headsep=.5in
\hoffset  -.75in

\pagestyle{empty}

\renewcommand{\thefootnote}{\fnsymbol{footnote}}
\begin{document}

\begin{center}
{\bf \huge{Computational Neuroscience Summer Program: Introductory Course}}\\
\large{May 31 -- June 3, 2011}
\end{center}

\setlength{\unitlength}{1in}

\begin{picture}(6,.1) 
\put(0,0) {\line(1,0){6.25}}         
\end{picture}

 

\renewcommand{\arraystretch}{2}
\begin{tabular}{lp{5in}}
\textbf{Instructors:}& Dr. Joshua Jacobs~(\texttt{joshua.jacobs@drexel.edu}) \\ 
                            & Dr. Jeremy Manning~(\texttt{manning3@mail.med.upenn.edu})\\
\textbf{Suggested texts:} & \textit{Theoretical Neuroscience}, Dayan and Abbott\\ 
                                       & \textit{Principles of Neural Science}, Kandel, Schwartz, and Jessell\\
                                       & \textit{Matlab for Neuroscientists}, Wallisch \textit{et al.}\\
\end{tabular}

\vskip.25in
\noindent \textbf{Course overview:} This intensive introductory course is intended to familiarize students with basic techniques in computational modeling and analysis of neural data using Matlab.  Students may (and are encouraged to) work together on assignments, but each student will be expected to hand in their own work.  Assignments will be reviewed, but no formal grades will be assigned.

\vskip.25in

\noindent \textbf{Course Outline:} 

\begin{center} \begin{minipage}{5.5in}
\begin{flushleft}
Orientation and ethics training \dotfill May 31 (AM)\\
Introduction to programming in Matlab \dotfill May 31 (PM)\\
Introduction to computational modeling \dotfill June 1 (AM)\\
Integrate-and-fire neuron model \dotfill June 1 (PM)\\
Hodgkin-Huxley neuron model \dotfill June 2 (AM)\\
Extensions of the Hodgkin-Huxley model \dotfill June 2 (PM)\\
Neural data processing techniques \dotfill June 3 (AM)\\
Open lab time \dotfill June 3 (PM)\\
\end{flushleft}
\end{minipage}
\end{center}

\vskip.25in
\noindent \textbf{Note:} The above course outline is approximate and is subject to change pending students' needs and interests. Because of the brief duration of this course, we are only able to provide a small ``taste'' of the diverse and evolving field of computational neuroscience.  Students seeking more in-depth coverage of computational neuroscience, including the topics discussed in this course, are encouraged to read the suggested texts.

\vspace*{.15in}


\end{document}